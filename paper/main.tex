\documentclass[11pt]{article}

% --- Packages ---
\usepackage[utf8]{inputenc}
\usepackage[T1]{fontenc}
\usepackage[margin=1in]{geometry}
\usepackage{graphicx}
\usepackage{booktabs}           % Professional tables
\usepackage{amsmath}
\usepackage{amssymb}
\usepackage{listings}           % Code listings
\usepackage{tikz}               % Architecture diagrams
\usepackage{pgfplots}           % Benchmark charts
\usepackage[ruled,vlined]{algorithm2e}  % Algorithm pseudocode
\usepackage{xcolor}
\usepackage{subcaption}         % Subfigures
\usepackage{url}
\usepackage{microtype}          % Improved typography
\usepackage{enumitem}           % List customization

% --- Hyperref (load last) ---
\usepackage[colorlinks=true,
            linkcolor=blue,
            citecolor=blue,
            urlcolor=blue]{hyperref}

% --- pgfplots compatibility ---
\pgfplotsset{compat=1.18}

% --- Code listing style ---
\lstdefinestyle{cstyle}{
    language=C,
    basicstyle=\ttfamily\small,
    keywordstyle=\color{blue}\bfseries,
    commentstyle=\color{gray}\itshape,
    stringstyle=\color{red},
    showstringspaces=false,
    numbers=left,
    numberstyle=\tiny\color{gray},
    breaklines=true,
    frame=single,
    captionpos=b,
    tabsize=4
}
\lstset{style=cstyle}

% --- Title and Author ---
\title{EdgeNN: A Quantization-First C Library for Deterministic \\
       DNN/RNN/Transformer Inference on ARM Microcontrollers}

\author{
    Dimitrios Kafetzis\textsuperscript{1,2} \\[4pt]
    \textsuperscript{1}DeepSea Technologies (Nabtesco subsidiary), IoT Department \\
    \textsuperscript{2}Athens University of Economics and Business, Department of Informatics \\[4pt]
    \texttt{dimitrioskafetzisd@gmail.com}
}

\date{\today}

% =================================================================
\begin{document}

\maketitle

% --- Abstract ---
\begin{abstract}
Deploying neural networks on ARM microcontrollers demands inference engines
that operate within tight memory budgets (64\,KB--2\,MB SRAM), deliver
deterministic latency for safety-critical applications, and avoid dynamic
memory allocation that introduces fragmentation and timing jitter.
%
Existing solutions either impose significant overhead (TensorFlow Lite
Micro, $\sim$100\,KB Flash), lack graph-level runtime capabilities
(CMSIS-NN), or require complex compiler toolchains unsuitable for
bare-metal deployment (microTVM).
%
Moreover, no current library provides unified operator coverage spanning
Dense/Convolutional, Recurrent (LSTM, GRU), and Transformer (Multi-Head
Attention) architectures---all within a zero-allocation execution model.

We present \textbf{EdgeNN}, a pure C11 library designed from the ground up
for deterministic neural network inference on ARM Cortex-M and Cortex-A
processors. EdgeNN employs a quantization-first design with INT8 symmetric
quantization, INT32 accumulation, and fixed-point requantization that
eliminates floating-point operations from the inference path. An arena-based
memory manager with a ping-pong buffer scheme provides zero-allocation
inference with bounded, predictable SRAM usage. EdgeNN implements 18
operators across DNN, RNN, and Transformer architectures, validated by 118
unit tests.

Preliminary evaluation on [target platform] demonstrates [X$\times$ speedup
/ Y\% Flash reduction / Z\% SRAM savings] compared to TensorFlow Lite
Micro, while maintaining less than 1 quantization step of accuracy
degradation. EdgeNN is open-source under the MIT License at
\url{https://github.com/Dimitrios-Kafetzis/EdgeNN}.
\end{abstract}

\textbf{Keywords:} edge inference, microcontroller, neural network,
quantization, ARM Cortex-M, LSTM, Transformer, embedded systems

% --- Sections ---
% =============================================================================
% Section 1: Introduction (~1.5 pages)
% =============================================================================

\section{Introduction}
\label{sec:introduction}

% --- Opening paragraph: The problem ---
% Edge AI inference on MCUs is critical for latency-sensitive, privacy-preserving,
% bandwidth-constrained applications. MCU constraints: 64KB-2MB SRAM, 256KB-2MB
% Flash, no MMU, often bare-metal or minimal RTOS. Need for deterministic,
% bounded-latency inference in safety-critical systems.
%
% TODO: Write opening paragraph

% --- Paragraph 2: Existing solutions and their limitations ---
% TFLite Micro, CMSIS-NN, microTVM, X-CUBE-AI, TinyEngine
%
% TODO: Write existing solutions paragraph

% --- Key gap statement ---
% TODO: Write gap statement

% --- Paragraph 3: Our contribution (EdgeNN) ---
% Enumerate 5-6 specific contributions:
% 1. Pure C11 library with zero dynamic allocation
% 2. Quantization-first design: INT8 symmetric, per-channel, INT16 for sensitive layers
% 3. Full operator coverage: DNN + RNN + Transformer
% 4. Layered HAL with generic C fallback and optimized backends
% 5. Static graph runtime with ping-pong buffer scheme
% 6. Model partitioning support for distributed edge inference
%
% TODO: Write contributions paragraph

% --- Paragraph 4: Paper organization ---
% TODO: Write organization paragraph

% Key references to cite:
% \cite{david2021tensorflow}  — TFLite Micro
% \cite{lai2018cmsis}         — CMSIS-NN
% \cite{lin2020mcunet}        — MCUNet
% \cite{lin2021mcunetv2}      — TinyEngine / MCUNetV2
% \cite{chen2018tvm}          — TVM / microTVM
% \cite{kafetzis2025wiopt}    — Your WiOpt 2025 paper

% =============================================================================
% Section 2: Related Work (~1.5 pages)
% =============================================================================

\section{Related Work}
\label{sec:related_work}

% ---------------------------------------------------------------------------
\subsection{Inference Frameworks for Microcontrollers}
\label{sec:related_frameworks}

TensorFlow Lite Micro (TFLite~Micro)~\cite{david2021tensorflow} is the most
widely deployed inference framework for microcontrollers.  It employs an
interpreter-based architecture in which a \texttt{MicroInterpreter} dispatches
operators at runtime from a registered kernel table.  While this design
provides flexibility and broad operator coverage for convolutional and dense
architectures, it introduces several limitations for resource-constrained
targets.  The interpreter runtime itself occupies approximately 100\,KB of
Flash, the C++ implementation requires exception handling support and
name-mangled symbols, and the built-in memory planner employs a
greedy algorithm that does not guarantee minimal allocation and can exhibit
non-deterministic allocation patterns across model versions.  Support for
recurrent operators (LSTM, GRU) is partial and limited to specific kernel
registrations, while Transformer operators such as Multi-Head Attention are
absent from the standard kernel set.

CMSIS-NN~\cite{lai2018cmsis}, developed by Arm, takes a fundamentally different
approach by providing hand-optimized operator kernels that exploit the DSP
extensions (SIMD MAC, saturating arithmetic) available on Cortex-M4/M7 cores.
Functions such as \texttt{arm\_fully\_connected\_s8()} and
\texttt{arm\_convolve\_s8()} achieve near-peak throughput for INT8 inference,
and the library footprint is minimal (approximately 5\,KB of Flash for a
typical operator subset).  However, CMSIS-NN operates strictly at the operator
level: it provides no graph runtime, no model loading mechanism, no layer
sequencing logic, and no support for RNN or Transformer architectures.
Integrating CMSIS-NN kernels into a complete inference pipeline requires
substantial application-level code for buffer management, operator scheduling,
and quantization parameter propagation.

The Apache TVM project~\cite{chen2018tvm} offers microTVM, which compiles
models from high-level frameworks into target-specific C code through an LLVM
backend.  This compiler-based approach can produce highly optimized kernels, but
the deployment pipeline requires an LLVM toolchain, a Python-based compilation
host, and runtime support for tensor allocation---making it poorly suited for
bare-metal environments.  Vendor-specific tools, including
STMicroelectronics' X-CUBE-AI and Edge Impulse, provide integrated model
conversion and deployment but are proprietary, locked to specific hardware
families, and offer no source-level access for customization.  Academic projects
such as TinyEngine~\cite{lin2021mcunetv2} have demonstrated that co-designing
the inference engine with the model architecture can yield substantial
efficiency gains, reducing peak memory through techniques such as patch-based
inference.  However, TinyEngine focuses exclusively on convolutional
architectures and is tightly coupled to the MCUNet model family.

% ---------------------------------------------------------------------------
\subsection{Quantization for MCU Inference}
\label{sec:related_quantization}

Quantization reduces the precision of weights and activations from 32-bit
floating-point to lower-bitwidth integer representations, enabling efficient
inference on processors without floating-point units.
Jacob et al.~\cite{jacob2018quantization} established the foundational
quantization scheme adopted by TensorFlow, in which real values are
approximated as $r = S(q - Z)$, where $S$ is a floating-point scale, $Z$ is an
integer zero point, and $q$ is the quantized integer value.  Two principal
approaches exist for determining quantization parameters: post-training
quantization (PTQ), which calibrates scales from a representative dataset after
training, and quantization-aware training (QAT), which simulates quantization
effects during the training loop to recover accuracy.

A key design choice is the quantization granularity.  Per-tensor quantization
assigns a single scale and zero point to an entire tensor, minimizing metadata
overhead but sacrificing accuracy when the dynamic range varies significantly
across output channels.  Per-channel quantization, which assigns independent
parameters to each output channel of a weight tensor, provides substantially
better accuracy for convolutional and dense layers at the cost of per-channel
requantization during inference.  Hybrid precision schemes that combine INT8 for
the majority of computations with INT16 for precision-sensitive operations have
proven valuable for recurrent architectures: LSTM cell state, which
accumulates over potentially hundreds of time steps, exhibits measurable
accuracy degradation when constrained to INT8 precision, while INT16
accumulation maintains error below 0.5\% over typical sequence lengths.

On ARM Cortex-M cores, quantized inference maps naturally to the DSP extension
instructions.  The \texttt{SMLAD} instruction performs dual signed
multiply-accumulate in a single cycle, and the \texttt{SSAT} instruction
provides saturating arithmetic for output clamping.  The requantization step---
converting INT32 accumulators back to INT8---is implemented as a fixed-point
multiply-shift operation: the real scale ratio is decomposed into a normalized
INT32 multiplier in $[2^{30}, 2^{31})$ and an integer right-shift, eliminating
all floating-point operations from the inference path.

% ---------------------------------------------------------------------------
\subsection{Neural Network Architectures on MCUs}
\label{sec:related_architectures}

The dominant architectures deployed on microcontrollers fall into three
families, each with distinct computational characteristics and memory access
patterns.

\textbf{Convolutional and Dense Networks.}
MobileNet and its variants employ depthwise separable convolutions to reduce
computation while maintaining accuracy, and have been widely deployed on
Cortex-M7 targets via TFLite~Micro.  MCUNet~\cite{lin2020mcunet} introduced
neural architecture search (NAS) specifically constrained to MCU memory
budgets, producing architectures that fit within 256\,KB of SRAM.
MCUNetV2~\cite{lin2021mcunetv2} further reduced peak memory through patch-based
inference, processing input images in spatial tiles.  The depthwise-separable
CNN (DS-CNN) architecture has become the standard for keyword spotting tasks,
achieving over 95\% accuracy on the Google Speech Commands dataset with fewer
than 25K parameters.

\textbf{Recurrent Networks.}
LSTM and GRU networks are essential for temporal sequence processing in
applications such as predictive maintenance, anomaly detection, and voice
activity detection.  Deploying RNNs on MCUs poses unique challenges: the
recurrent state must persist across time steps, gate computations involve
multiple matrix multiplications per step, and the cell state in LSTM is
sensitive to quantization error that accumulates over the sequence length.
FastGRNN~\cite{kusupati2018fastgrnn} addressed the size constraint by
introducing low-rank and sparse gate matrices, reducing model footprint to
kilobyte scale.  However, the inference runtime for FastGRNN was implemented as
a standalone C function rather than an integrated library, and no existing
general-purpose MCU framework provides well-optimized, quantized LSTM and GRU
operators with proper INT16 cell state handling.

\textbf{Transformer Networks.}
The attention mechanism has achieved state-of-the-art results across NLP and
increasingly in vision and time-series tasks.  EdgeBERT~\cite{tambe2021edgebert}
demonstrated that sentence-level energy optimization and early exit strategies
can make BERT-class models feasible on edge processors, though its target
was FPGA accelerators rather than general-purpose MCUs.  Deploying Multi-Head
Attention on Cortex-M cores is challenging due to the quadratic memory
requirement of the attention score matrix ($O(\text{seq}^2)$) and the need for
softmax computation in fixed-point arithmetic.  No existing MCU inference
library provides native Transformer operator support: TFLite~Micro lacks
attention operators, CMSIS-NN provides only DNN kernels, and microTVM requires
model-specific compilation.

This fragmentation means that practitioners deploying multi-architecture
systems---for example, a CNN feature extractor followed by an LSTM temporal
model, or a Transformer encoder for sensor fusion---must integrate multiple
libraries or write custom operator code, with no unified framework for memory
management, quantization, or execution.

% ---------------------------------------------------------------------------
\subsection{Model Partitioning and Distributed Edge Inference}
\label{sec:related_partitioning}

When a single MCU lacks sufficient memory or compute capacity for a complete
model, partitioning the model across multiple devices offers a viable
alternative to model compression.  Kang et al.~\cite{kang2017neurosurgeon}
introduced Neurosurgeon, which dynamically selects the optimal partition point
between a mobile device and the cloud by profiling per-layer latency and data
transfer cost.  Subsequent work extended this idea to edge-only settings where
multiple resource-constrained devices collaborate without cloud connectivity.

Kafetzis and Koutsopoulos~\cite{kafetzis2025wiopt} addressed the specific
problem of DNN partitioning for resource-constrained wireless edge networks,
formulating the partition point selection as an optimization problem that
jointly considers computation latency, communication bandwidth, and per-device
memory constraints.  This work is directly relevant to MCU deployments where
heterogeneous devices (e.g., an STM32MP1 with both Cortex-A7 and Cortex-M4
cores) can split inference across processing elements connected via SPI or
shared memory.  Extending partitioning to multi-layer Transformer architectures
introduces additional complexity, as attention layers exhibit data-dependent
communication volumes (due to variable sequence lengths) and benefit from
speculative execution of subsequent layers while communication is in progress.

Critically, partitioning requires the inference library to support clean layer
boundary semantics: the output tensor of one partition must carry sufficient
metadata (shape, data type, quantization parameters) for the receiving
partition to continue execution without ambiguity.  Existing frameworks do not
explicitly design for this requirement---TFLite~Micro's internal tensor
representation is tightly coupled to its interpreter state, and CMSIS-NN
operates on raw buffers without structured metadata.

% ---------------------------------------------------------------------------
\vspace{0.5em}
\noindent\textbf{Summary.}
Table~\ref{tab:related_comparison} in Section~\ref{sec:evaluation} provides a
detailed feature comparison.  In summary, existing MCU inference tools occupy
distinct but non-overlapping niches: TFLite~Micro provides a graph runtime but
with substantial overhead and limited architecture coverage; CMSIS-NN delivers
optimized kernels but without runtime infrastructure; microTVM offers a
compiler pipeline but with deployment complexity; and vendor tools sacrifice
portability for integration convenience.  No existing library combines
(i)~zero-allocation deterministic inference, (ii)~unified DNN, RNN, and
Transformer operator coverage, (iii)~a quantization-first INT8/INT16 design
with fixed-point requantization, (iv)~a portable HAL with optimized backend
support, and (v)~partitioning-aware tensor descriptors---within a pure C11
implementation under 50\,KB of Flash.  EdgeNN is designed to fill this gap.

% =============================================================================
% Section 3: Design Principles (~1 page)
% =============================================================================

\section{Design Principles}
\label{sec:design_principles}

% Present 5 core principles with justification.

% ---------------------------------------------------------------------------
\subsection{P1: Zero-Allocation Inference}
\label{sec:p1_zero_alloc}

% Problem: malloc/free introduces jitter (15-40us per call), fragmentation,
% non-deterministic execution time.
% Solution: All memory pre-planned at model load time; inference uses only
% pre-allocated arena buffers.
% Quantify: Deterministic cycle count variance (<1%) vs TFLite Micro.
%
% TODO: Write P1

% ---------------------------------------------------------------------------
\subsection{P2: Tensor Descriptor, Not Object}
\label{sec:p2_tensor_descriptor}

% Problem: Object-oriented tensor abstractions carry overhead.
% Solution: Lightweight 80-byte struct with pointer + shape + strides + qparams.
% Benefit: Zero-cost abstraction.
%
% TODO: Write P2

% ---------------------------------------------------------------------------
\subsection{P3: Quantization-First Design}
\label{sec:p3_quantization_first}

% Problem: Retrofitting quantization onto float-first libraries is suboptimal.
% Solution: INT8 is primary; FP32 is fallback for reference/validation.
% INT16 for accumulation-sensitive ops (LSTM cell state, attention scores).
% Fixed-point multiplier+shift for requantization without runtime float.
%
% TODO: Write P3

% ---------------------------------------------------------------------------
\subsection{P4: Layered HAL with Generic Fallback}
\label{sec:p4_layered_hal}

% Problem: ARM-optimized intrinsics are not portable; pure C is slow.
% Solution: Every operator has generic C + ifdef optimized backends.
% Supports: CMSIS-NN, Helium/MVE, NEON at compile time.
%
% TODO: Write P4

% ---------------------------------------------------------------------------
\subsection{P5: Modular Architecture with Dead Code Elimination}
\label{sec:p5_modular}

% Problem: TFLite Micro's monolithic interpreter pulls in unused code.
% Solution: LTO with -ffunction-sections -fdata-sections -Wl,--gc-sections.
% Only operators used in the model are included in the final binary.
%
% TODO: Write P5

% =============================================================================
% Section 4: System Architecture (~2 pages)
% =============================================================================

\section{System Architecture}
\label{sec:architecture}

% ---------------------------------------------------------------------------
\subsection{Architecture Overview}
\label{sec:arch_overview}

% Present the 6-layer stack diagram (Figure 1).
% Explain layer dependencies and the flow from user API to hardware.
%
% TODO: Write overview and reference architecture_stack figure

% ---------------------------------------------------------------------------
\subsection{Memory Management}
\label{sec:arch_memory}

% Arena allocator: bump pointer, O(1) allocation, zero fragmentation.
% Ping-pong buffer scheme: two buffers alternate between layers.
% Memory planning: dry-run at model load for peak scratch, weight memory.
% Diagram: Flash zones vs SRAM zones.
% Memory budget analysis for three example models.
%
% TODO: Write memory management subsection
% TODO: Reference arena_pingpong figure

% ---------------------------------------------------------------------------
\subsection{Tensor Descriptor Design}
\label{sec:arch_tensor}

% Show the struct definition (code listing).
% Explain quantization parameters (per-tensor, per-channel, fixed-point).
% Layout support (NHWC, NCHW, NC, NTC) and stride computation.
%
% TODO: Write tensor descriptor subsection

% ---------------------------------------------------------------------------
\subsection{Static Graph Runtime}
\label{sec:arch_runtime}

% Layer descriptor array with op_type, params pointer, tensor indices.
% Sequential execution with kernel dispatch.
% Profiling hooks (cycle counter per layer).
% Comparison with TFLite Micro's interpreter dispatch.
%
% TODO: Write runtime subsection

% =============================================================================
% Section 5: Operator Implementation (~2 pages)
% =============================================================================

\section{Operator Implementation}
\label{sec:operators}

% ---------------------------------------------------------------------------
\subsection{DNN Operators}
\label{sec:ops_dnn}

% Dense: INT8 matmul with INT32 accumulation, per-channel requantization,
%        fused activation.
% Conv2D: im2col + GEMM strategy, 1x1 pointwise optimization, SAME/VALID.
% Depthwise Conv2D: channel-wise direct convolution (no im2col).
% Pooling: MaxPool (dtype-preserving), AvgPool (with requantization).
% Activations: ReLU/ReLU6 as clamping, Sigmoid/Tanh/GELU via 256-entry LUTs.
%
% TODO: Write DNN operators subsection
% TODO: Reference operator_support table

% ---------------------------------------------------------------------------
\subsection{RNN Operators}
\label{sec:ops_rnn}

% LSTM: Fused gate computation (single matmul for all 4 gates),
%       INT8/INT16 hybrid (gates INT8, cell state INT16).
% GRU: 3-gate with reset-gate-in-hidden-path optimization.
% Sequence unrolling: state carry-over, bidirectional wrapper.
%
% TODO: Write RNN operators subsection

% ---------------------------------------------------------------------------
\subsection{Transformer Operators}
\label{sec:ops_transformer}

% Multi-Head Attention: Q/K/V projection, per-head scaled dot-product,
%   INT16 score accumulation, softmax in fixed-point, optional KV-cache.
% Layer Normalization: INT32 accumulation for mean/variance,
%   fixed-point inverse sqrt.
% FFN: Two Dense layers with GELU activation (LUT-based).
% Positional Encoding: Precomputed sinusoidal tables, RoPE planned.
%
% TODO: Write Transformer operators subsection

% ---------------------------------------------------------------------------
\subsection{Key Implementation Details}
\label{sec:ops_details}

% LUT generation: building 256-entry lookup tables for nonlinear activations.
% Fused operators: Conv+BN+ReLU at weight-folding level.
% INT16 precision decision: INT8 cell state causes >5% accuracy loss in LSTM
%   after 100 time steps, INT16 maintains <0.5% error.
%
% TODO: Write implementation details subsection

% =============================================================================
% Section 6: Quantization Pipeline (~1 page)
% =============================================================================

\section{Quantization Pipeline}
\label{sec:quantization}

% ---------------------------------------------------------------------------
\subsection{Quantization Scheme}
\label{sec:quant_scheme}

% Symmetric INT8 for weights (zero_point = 0).
% Asymmetric or symmetric INT8 for activations.
% Per-channel quantization for convolution and dense weights.
% Compatibility with TFLite QAT exports.
%
% TODO: Write quantization scheme subsection
% TODO: Reference quantization_pipeline figure

% ---------------------------------------------------------------------------
\subsection{Fixed-Point Requantization}
\label{sec:quant_requantize}

% The multiplier+shift decomposition:
%   real_mult = (in_scale * weight_scale) / out_scale
% Decompose via frexp() into normalized significand [0.5, 1.0) -> INT32 + exponent.
% Runtime operation: result = (accumulator * multiplier) >> 31 >> shift + zero_point.
% No floating-point in INT8 inference path.
%
% TODO: Write requantization subsection
% Consider including Algorithm environment with pseudocode

% ---------------------------------------------------------------------------
\subsection{LUT-Based Activation Functions}
\label{sec:quant_lut}

% For each quantized activation (sigmoid, tanh, GELU), precompute 256-entry LUT
% at model load time.
% LUT maps every INT8 input to INT8 output: O(1) activation, 256 bytes memory.
% Accuracy comparison: LUT vs full-precision for sigmoid/tanh.
%
% TODO: Write LUT subsection
% TODO: Reference benchmark_accuracy figure/table

% =============================================================================
% Section 7: Evaluation (~2 pages)
% THIS IS THE MOST IMPORTANT SECTION.
% =============================================================================

\section{Evaluation}
\label{sec:evaluation}

% ---------------------------------------------------------------------------
\subsection{Experimental Setup}
\label{sec:eval_setup}

% Hardware: STM32H7 (Cortex-M7, 480MHz, 1MB SRAM, 2MB Flash),
%           NXP i.MX RT1060 (Cortex-M7, 600MHz, 1MB SRAM)
% Compiler: arm-none-eabi-gcc 12.x, optimization -Os -flto
% Measurement: DWT cycle counter (hardware), averaged over 1000 inferences
% Baselines: TFLite Micro 2.x, CMSIS-NN 5.x
%
% TODO: Write experimental setup

% ---------------------------------------------------------------------------
\subsection{Benchmark Models}
\label{sec:eval_models}

% DS-CNN:        3x DWConv2D + Dense, ~25K params, keyword spotting
% MobileNet-v2:  Conv + DWConv + Dense, ~200K params, image classification
% LSTM-AE:       2-layer LSTM autoencoder, ~50K params, anomaly detection
% TinyBERT-2L:   2-layer Transformer d=64, ~150K params, text classification
%
% TODO: Write benchmark models table
% TODO: Reference hardware_targets table

% ---------------------------------------------------------------------------
\subsection{Latency Results}
\label{sec:eval_latency}

% Table: Model | EdgeNN (ms) | TFLite Micro (ms) | CMSIS-NN (ms) | Speedup
% Bar chart: benchmark_latency figure
%
% NOTE: Actual benchmarks must be measured on real hardware.
% For initial preprint, can use host benchmarks or QEMU emulation.
%
% TODO: Fill in benchmark numbers
% TODO: Reference benchmark_latency figure and benchmark_results table

% ---------------------------------------------------------------------------
\subsection{Memory Footprint}
\label{sec:eval_memory}

% Table: Model | EdgeNN Flash | TFLite Flash | EdgeNN SRAM | TFLite SRAM
% Key metric: library core Flash overhead (<20KB vs ~100KB)
%
% TODO: Fill in memory numbers
% TODO: Reference benchmark_memory figure

% ---------------------------------------------------------------------------
\subsection{Accuracy}
\label{sec:eval_accuracy}

% Table: Model | FP32 Ref (%) | EdgeNN INT8 (%) | TFLite INT8 (%) | SQNR (dB)
%
% TODO: Fill in accuracy numbers
% TODO: Reference benchmark_accuracy figure

% ---------------------------------------------------------------------------
\subsection{Ablation Studies}
\label{sec:eval_ablation}

% 1. INT16 LSTM cell state vs INT8-only (accuracy over time steps)
% 2. LUT vs polynomial activation approximation (accuracy vs cycles)
% 3. Ping-pong buffer vs separate allocation (peak SRAM)
% 4. LTO impact on Flash footprint (with vs without)
%
% TODO: Write ablation studies

% =============================================================================
% Section 8: Model Partitioning Support (~1 page)
% This is the unique research angle differentiating EdgeNN from competitors.
% =============================================================================

\section{Model Partitioning Support}
\label{sec:partitioning}

% ---------------------------------------------------------------------------
\subsection{Motivation}
\label{sec:part_motivation}

% Single MCU may not have enough SRAM/Flash for larger models.
% Heterogeneous edge deployments (e.g., STM32MP1: Cortex-A7 + Cortex-M4).
% Your WiOpt 2025 research: dynamic DNN partitioning for resource-constrained
% wireless edge.
%
% TODO: Write motivation subsection

% ---------------------------------------------------------------------------
\subsection{Partitioning-Aware Design}
\label{sec:part_design}

% Modular graph runtime: layers can be split at any boundary.
% Tensor descriptors carry full quantization info -> zero-ambiguity at
% partition boundaries.
% Scratch arena isolation: each partition manages its own scratch.
% Communication protocol: serialize intermediate tensor (data + descriptor)
% over SPI/UART/network.
%
% TODO: Write design subsection

% ---------------------------------------------------------------------------
\subsection{Preliminary Demonstration}
\label{sec:part_demo}

% Example: Split 4-layer model between two STM32 boards via SPI.
% Device A: layers 1-2, serialize intermediate, send via SPI.
% Device B: receive, layers 3-4, produce output.
% Latency breakdown: compute vs communication.
% Reference ACM MobiSys 2026 work on multi-layer Transformer partitioning
% with speculative execution.
%
% TODO: Write demonstration subsection

% =============================================================================
% Section 9: Conclusion and Future Work (~0.5 page)
% =============================================================================

\section{Conclusion}
\label{sec:conclusion}

% --- Conclusion ---
% Summarize EdgeNN's contributions.
% Key results: X* speedup, Y% less Flash, deterministic execution.
% Open-source availability: GitHub link.
%
% TODO: Write conclusion paragraph

% --- Future Work ---
\subsection*{Future Work}

% - INT4 weight-only quantization
% - Helium/MVE kernels for Cortex-M55
% - Operator fusion (Conv+BN+ReLU, LayerNorm+Dense)
% - Full Python model converter with quantization calibration
% - Integration with speculative execution for partitioned Transformer inference
% - MISRA-C compliance for automotive/safety certification path
%
% TODO: Write future work paragraph


% --- References ---
\bibliographystyle{plain}
\bibliography{references}

% --- Appendix ---
\appendix
% =============================================================================
% Appendix
% =============================================================================

\section{Operator API Reference}
\label{sec:appendix_api}

% Optional: Include key struct definitions, operator signatures, or
% additional benchmark data that does not fit in the main body.
%
% TODO: Add appendix content if needed


\end{document}
