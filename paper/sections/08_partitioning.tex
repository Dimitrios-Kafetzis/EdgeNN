% =============================================================================
% Section 8: Model Partitioning Support (~1 page)
% This is the unique research angle differentiating EdgeNN from competitors.
% =============================================================================

\section{Model Partitioning Support}
\label{sec:partitioning}

% ---------------------------------------------------------------------------
\subsection{Motivation}
\label{sec:part_motivation}

% Single MCU may not have enough SRAM/Flash for larger models.
% Heterogeneous edge deployments (e.g., STM32MP1: Cortex-A7 + Cortex-M4).
% Your WiOpt 2025 research: dynamic DNN partitioning for resource-constrained
% wireless edge.
%
% TODO: Write motivation subsection

% ---------------------------------------------------------------------------
\subsection{Partitioning-Aware Design}
\label{sec:part_design}

% Modular graph runtime: layers can be split at any boundary.
% Tensor descriptors carry full quantization info -> zero-ambiguity at
% partition boundaries.
% Scratch arena isolation: each partition manages its own scratch.
% Communication protocol: serialize intermediate tensor (data + descriptor)
% over SPI/UART/network.
%
% TODO: Write design subsection

% ---------------------------------------------------------------------------
\subsection{Preliminary Demonstration}
\label{sec:part_demo}

% Example: Split 4-layer model between two STM32 boards via SPI.
% Device A: layers 1-2, serialize intermediate, send via SPI.
% Device B: receive, layers 3-4, produce output.
% Latency breakdown: compute vs communication.
% Reference ACM MobiSys 2026 work on multi-layer Transformer partitioning
% with speculative execution.
%
% TODO: Write demonstration subsection
